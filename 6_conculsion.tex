\section{Conclusions}
\label{section:conclusions}
\subsection{Summary of Research Findings}
Although existing long-term aircraft maintenance planning studies have considered scheduling decisions, none of them have considered aircraft-to-station assignment, which remains a concern for a major airline with a large fleet of heterogeneous aircraft and a set of spatially distributed maintenance facilities. 
Motivated by this important research gap, we presented an integrated aircraft maintenance planning model over an extended planning horizon.  Given that directly solving the integrated optimization problem with a commercial integer programming solver is impractical for extended planning periods (e.g., six months) or under tight capacity constraints, we proposed two decomposition-based heuristics. The first solution approach separates assignment decisions from scheduling decisions, while considering the effect of scheduling decisions on assignment decisions. The second approach addresses scalability by introducing a temporal decomposition strategy that divides the extended planning horizon into partially overlapping time horizons. To mitigate the infeasibility and sub-optimality issues inherent in temporal decomposition, we carefully selected the time horizon and rolling period parameters using a grid search.
 
With real-world data support from a major U.S.-based airline, we first solved the joint optimization problem, when the planning horizon is short, with a commercial solver under a full-capacity scenario and then examined the impact of capacity reductions on aircraft maintenance planning decisions. Further, we evaluated the performance of two proposed solution approaches by comparing them with directly solving the joint optimization problem with a commercial solver. Our results show that both proposed solution approaches (i.e., SH and TD) can reduce the computational time by more than 80\%, with only minimal degradation in the optimization objective (no more than 2\%). Such small differences in the objective value are also validated by examining the absolute and relative schedule deviations. We find that the schedule deviations are zero for most of the tuples, although substantial deviations can be observed for a very small percentage of tuples. Therefore, we conclude that both SH and TD can strike preferable trade-offs between solution time and quality.

Due to the highly satisfactory performance of the proposed solution approaches, we were able to conduct a long-term analysis spanning 180 days. One notable finding is that our proposed optimization method can schedule a tuple for maintenance over a long planning horizon multiple times, which is a key feature desired by the anonymous airline while being missed in the literature \citep{yan2008long}. Finally, we demonstrated the practical relevance of such a long-term aircraft maintenance planning model by analyzing the impact of station access reductions during anticipated travel peaks.


\subsection{Practice Significance of this Study}
Our research was conducted in response to the practical needs of AA. The current long-term aircraft maintenance planning practice is largely manual and based on planners’ experience, similar to what was described by \cite{boere1977air}. To replace the time-consuming and labor-extensive process, internal consultants developed some integer programming models; however, one of the major shortcomings was that over a planning horizon, a single check can be scheduled for each aircraft, while it is highly desirable for AA that a series of checks could be scheduled over time. With that clear research need to be addressed, we presented a joint optimization model and two effective decomposition-based solution approaches, as described in earlier sections of this paper. Therefore, this study is expected to advance the current practice in at least two ways: first, it can replace the current labor-extensive aircraft maintenance planning process and yield more favorable maintenance plans; second, it enhances the current optimization methods developed by the internal consultants at AA and thus becomes a cornerstone of their next-generation computer-based aircraft maintenance scheduling system.



\subsection{Research Extensions}
The current research presented in this paper can be improved or extended in multiple ways, including:
\begin{enumerate}
    \item In a long-term planning problem, the inherent uncertainty about maintenance station capabilities or capacities should be considered \citep{sun2015stochastic}. The possibility of rescheduling in anticipation of flight cancellations or other disruptions should be incorporated. 
    \item In addition to managing aircraft maintenance demand, maintenance capacity planning can be conducted at the same time. For instance, the timing of a major renovation for a maintenance station can be carefully selected to avoid major impacts on aircraft maintenance schedules. 
    \item The training of maintenance personnel and consideration of the licensing processes for engineers could be integrated to ensure that workforce readiness aligns with the fluctuating demand of maintenance.
\end{enumerate}

