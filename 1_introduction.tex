\section{Introduction} 
Aviation is one of the safest means of transportation compared to other modes, primarily due to the prioritization of safety by airlines over other key performance indicators \citep{chen2024advancing}. Aircraft operations in civil aviation are meticulously regulated by government agencies, such as the Federal Aviation Administration (FAA) in the U.S. and the European Union Aviation Safety Agency (EASA), among others. Violating safety regulations and policies imposed by those aviation authorities leads to the grounding of aircraft and other more severe penalties. All major airlines have thus established their aircraft maintenance programs to ensure each aircraft is safe to operate and meets all applicable regulatory requirements. While aircraft safety is essential, it comes with a significant financial burden, as maintaining aircraft is known to be very expensive.
Based on cost data collected from airlines across the globe, the International Air Transport Association (IATA) estimated that airlines worldwide spent over \$76 billion on Aircraft Maintenance, Repair, and Overhaul (MRO) in 2022 only, which accounted for nearly 11\% of the total airline operating cost \citep{iata2022mcx}. By contrast, the IATA projected that the airline net profit margin would improve to 2.7\% in 2024, a slight increase over 2.6\% in 2023 \citep{iata2023pressrelease}.
Consequently, while aircraft maintenance is crucial, it is also costly, necessitating proper planning to minimize costs and improve airline profitability while ensuring compliance with all regulatory requirements.

Many transportation and operations researchers have contributed to the critical research area of aircraft maintenance optimization. A comprehensive review of the literature, provided in Section~\ref{section:liteReview}, reveals that most existing studies focus on the aircraft maintenance routing problem. By contrast, only a handful have addressed how a fleet of aircraft should be maintained over an extended planning horizon, namely the long-term aircraft maintenance planning problem.
Key distinctions exist between these two major groups of studies (i.e., aircraft maintenance routing vs long-term aircraft maintenance planning) in terms of the planning period and the nature of maintenance activities analyzed. 
Regarding the planning horizon length, aircraft maintenance routing problems involve a short planning period (such as a week), whereas long-term aircraft maintenance problems span over multiple months or years. Regarding the nature of maintenance activities, short-term aircraft maintenance routing problems focus on routine and minor checks \citep{sriram2003optimization,liang2009aircraft}, while long-term aircraft maintenance problem addresses major checks such as C checks or D checks \citep{van2013aircraft}.
Given the practical need of a collaborating airline (which remains anonymous), we focus on the long-term aircraft maintenance planning problem in this paper.

Considering the existing studies on the long-term aircraft maintenance problem, we have identified three major shortcomings through a detailed literature review in Section~\ref{section:liteReview}.
First, existing studies in this area have focused mainly on the scheduling of aircraft for maintenance checks. As a result, only scheduling decisions have been optimized. For instance, \cite{deng2020practical} optimized long-term maintenance scheduling for A and C checks over a three-year period at a single maintenance station.  However, in the real-world practice for a major airline with a network of maintenance stations, long-term planning often requires both the scheduling of aircraft and the assignment to different maintenance stations. Jointly optimizing these decisions (i.e., scheduling and station assignment) remains crucial for effective maintenance planning, which should be done to fill an important gap. 
Second, we observe that most existing studies have considered only basic maintenance-related constraints, such as man-hour limits, while neglecting many other practical constraints. For instance, these studies failed to account for station-specific limits on the number of C or A checks (referred to as C check and A check limits, respectively), aircraft rotation requirement, station access limit, and the possibility of conducting multiple checks for a single aircraft over a planning period.
Third, the problem sizes addressed in existing studies on the long-term planning problem have been limited. While small-scale problems can be solved directly using commercial solvers, this direct solution approach becomes impractical as the problem size increases. To address this, some researchers have developed heuristic solution approaches. However, the quality of solutions produced by these heuristic approaches is unclear.


To address the research gaps identified above, we first develop a long-term aircraft maintenance optimization model that determines when and where an aircraft should be scheduled for maintenance. Many practical constraints specified by the collaborating U.S. airline (namely American Airlines or AA) have been considered, such as two C checks in the same group could be conducted on an aircraft on the same day while C checks from different groups cannot be mixed. In addition to directly solving the integrated optimization model by a commercial solver, two decomposition-based solution approaches are proposed, namely a scheduling-first-assignment-second approach and a temporal decomposition approach. Extensive computational experiments are conducted based on real-world data to validate the effectiveness, efficiency, and practical usefulness of the long-term aircraft maintenance planning method.



This paper has made several major contributions. The first contribution is from the modeling perspective. This paper introduces the first integrated optimization model for aircraft maintenance scheduling and station assignment, involving a heterogeneous fleet of aircraft, a network of maintenance bases with different capabilities and capacities, and covering an extended planning horizon. The second contribution lies in the design of effective solution algorithms. This study proposes two novel decomposition-based solution approaches that can achieve a near-optimal solution with a relative gap of below 2\% while reducing computation time by up to 80\% compared to directly solving the mixed integer program with a commercial solver. Finally, this paper presents a real-world aircraft maintenance planning case study that is larger in scale than any others in the current literature, measured by the fleet size, the number of maintenance facilities, and the number of maintenance capacity parameters. 




The remainder of this paper is described as follows. In Section~\ref{section:liteReview}, we review the literature on aircraft maintenance planning in civil aviation. In Section~\ref{Problem_definition_formulation}, we introduce the mixed integer programming formulation for the integrated aircraft maintenance scheduling and assignment problem, aiming to minimize the discounted maintenance costs over the planning horizon. In Section~\ref{section:solutionmethodology}, we present two different decomposition-based solution approaches to reduce the computation time. Case studies are conducted based on data from a U.S.-based airline, as described in Section~\ref{section:case_studies}. Finally, the findings and recommendations for future research directions are provided in Section~\ref{section:conclusions}.



